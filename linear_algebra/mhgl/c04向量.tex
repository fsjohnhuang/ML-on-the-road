\documentclass[12pt]{article}
\usepackage{amsmath}
\usepackage{fullpage}
\usepackage{CJKutf8}

\begin{document}
\begin{CJK}{UTF8}{gbsn}
\section{向量的意思}
向量也是一种矩阵.分别有四种意思:
\begin{itemize}
\item 平面坐标中的点
\item 从坐标原点$(0, 0)$出发到某个对应点的箭头
\item 多个箭头之和
\item 不以坐标原点出发到某个对应点的箭头
\end{itemize}
\paragraph{向量表示点,线,面}
\begin{equation}
\text{y轴上的点表示为:}\ c\left[\begin{array}{c}
0\\
1
\end{array}\right]|\ \forall c \in R
\end{equation}
\begin{equation}
\text{y轴表示为:}\ \left\lbrace \left. c\left[\begin{array}{c}
0\\
1
\end{array}\right]\right|\ \forall c \in R \right\rbrace
\end{equation}
\begin{equation}
\text{xy平面表示为:}\ \left\lbrace \left. c_1\left[\begin{array}{c}
0\\
1
\end{array}\right]+c_2\left[\begin{array}{c}
1\\
0
\end{array}\right]\right|\ \forall c_1,c_2 \in R \right\rbrace
\text{或}
\ \left\lbrace \left. c_1\left[\begin{array}{c}
3\\
1
\end{array}\right]+c_2\left[\begin{array}{c}
1\\
2
\end{array}\right]\right|\ \forall c_1,c_2 \in R \right\rbrace
\end{equation}
\section{向量的乘法}
\begin{equation}
\left[
\begin{array}{cc}
x_{11} & x_{12}
\end{array}
\right]
\left[
\begin{array}{c}
y_{11} \\
y_{21}
\end{array}
\right]
=
\left[
\begin{array}{c}
x_{11}y_{11}+x_{12}y_{21}
\end{array}
\right]
\end{equation}
\begin{equation}
\left[
\begin{array}{c}
x_{11} \\
x_{21}
\end{array}
\right]
\left[
\begin{array}{cc}
y_{11} & y_{12}
\end{array}
\right]
=
\left[
\begin{array}{cc}
x_{11}y_{11} & x_{11}y_{12} \\
x_{21}y_{11} & x_{21}y_{12} 
\end{array}
\right]
\end{equation}
和矩阵乘法一样不遵循乘法交换率.
\section{线性相关,线性无关}
对于下面方程式组
\begin{equation}
\left[
\begin{array}{c}
0\\\vdots\\0
\end{array}
\right]=c_1\left[
\begin{array}{c}
a_{11}\\\vdots\\a_{m1}
\end{array}
\right]
+\cdots+
c_n\left[
\begin{array}{c}
a_{1n}\\\vdots\\a_{mn}
\end{array}
\right]
\end{equation}
若$[c_1 \cdots c_n]^T$有且仅有一个解时,则称向量$[a_{11} \cdots a_{1n}]^T$到向量$[a_{11} \cdots a_{1n}]^T$为线性无关(线性独立);否则称为线性相关.
\section{线性子空间}
若集合$R^n$的子集$W$符合以下两个条件,那么称集合$W$是集合$R^n$的\textbf{线性子空间},简称\textbf{子空间}.
\begin{enumerate}
\item $[a_1\cdots a_n]^T\in W \Rightarrow c[a_1\cdots a_n]^T\in W\ |\ \forall c \in R$
\item $[a_1\cdots a_n]^T, [b_1\cdots b_n]^T\in W \Rightarrow [a_1+b_1\cdots a_n+b_n]^T\in W$
\end{enumerate}
具体就是\textbf{通过原点的线}和\textbf{通过原点的面}等,总之就是\textbf{通过原点}.
\paragraph{由向量空间生成的子空间}
\begin{equation}
C=\lbrace c_1[a_{11}\cdots a_{m1}]^T + \cdots + c_n[a_{1n}\cdots a_{mn}]^T\ |\ \forall c_1,\cdots,c_n \in R,\ [a_{11}\cdots a_{m1}]^T,\cdots,[a_{1n}\cdots a_{mn}]^T \in R^m\rbrace
\end{equation}
集合$C$就是由向量$[a_{11}\cdots a_{m1}]^T$到$a_{1n}\cdots a_{mn}]^T$生成的子空间.
\subsection{基,维度}
假设集合$W$为$R^m$的子空间,若满足以下三个条件:
\begin{enumerate}
\item $[a_{11} \cdots a_{m1}]^T, \cdots, [a_{1n} \cdots a_{mn}]^T \in R^m$
\item $[a_{11} \cdots a_{m1}]^T, \cdots, [a_{1n} \cdots a_{mn}]^T$为线性无关向量
\item $W=\lbrace c_1[a_{11} \cdots a_{m1}]^T + \cdots + c_n[a_{1n} \cdots a_{mn}]^T\ |\ \forall c_1,\cdots,c_n\in R \rbrace$
\end{enumerate}
那么我们称集合$\lbrace [a_{11} \cdots a_{m1}]^T, \cdots ,[a_{1n} \cdots a_{mn}]^T \rbrace$为"子空间$W$的基",而基的元素个数$n$称为"子空间$W$的维度",表示为$dimW$.
\subsubsection{事例}
定义$xy$平面为$W$,而$W$为$R^3$的子空间.向量$[3\ 1\ 0]^T$和$[1\ 2\ 0]^T$属于$R^3$,且为线性无关向量.\\显然$W=\lbrace c_1[3\ 1\ 0]^T  + c_2[1\ 2\ 0]^T\ |\ \forall c_1,\cdots,c_2\in R \rbrace$,因此集合$\lbrace [3\ 1\ 0]^T, [1\ 2\ 0]^T \rbrace$为"子空间$W$的基",$dimW$为$2$.



\end{CJK}
\end{document}