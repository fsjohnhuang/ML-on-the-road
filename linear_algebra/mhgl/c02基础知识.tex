\documentclass[12pt]{article}
\usepackage{amsmath}
\usepackage{CJKutf8}
\usepackage{fullpage}

\begin{document}
\begin{CJK}{UTF8}{gbsn}
\section{数的分类}
\begin{itemize}
\item[\textbf{复数}] 形如$\lbrace a+bi$ $|\ \forall a,\ b \in \text{实数},\ i\text{为虚数单位,定义为}i^2=-i\ \rbrace$.\\复数包含实数和纯虚数.
	\begin{itemize}
	\item[\textbf{实数}] 包含有理数(可以用$\dfrac{q}{p}$表示,其中$p\neq 0$且$q$为整数)和无理数
		\begin{itemize}
		\item[\textbf{整数有理数}] 正整数,$0$和负整数
		\item[\textbf{非整数有理数}] 有限小数(如$0.3$)和无限循环小数(如$0.333\dots$)
		\item[\textbf{无理数}] 无限不循环小数(如$\pi\text{和}\sqrt{2}$)
		\end{itemize}
	\item[\textbf{纯虚数}] 形如$\lbrace\ bi\ |\ b\text{为非零实数}\  \rbrace$
	\end{itemize}
\end{itemize}
\paragraph{虚数单位$i$的理解:}
$$x^2+5=0 \Rightarrow x^2-(-5)=0 \Rightarrow (x+\sqrt{5}i)(x-\sqrt{5}i)=0$$
$i$就是为形如$x^2+5=0$的方程而设想出来的数.上述方程中当$i^2=-1$时有解.

\section{命题}
可以判断其正确与否,且判断结果不随个人而改变的主张.\\
\paragraph{事例}
\begin{itemize}
\item[命题] 肥仔John是男的
\item[非命题] 肥仔John很健谈
\end{itemize}

\section{必要条件和充分条件}
如果$P$成立,那么$Q$就成立.那么$Q$为$P$的必要条件,而$P$为$Q$的充分条件.\\
数学符号为:
$$P \Rightarrow Q$$
如果$P$成立,那么$Q$就成立;如果$Q$成立,那么$P$就成立;这两个命题均正确,那么$P$与$Q$互为充分必要条件.\\
数学符号为:
$$P \Leftrightarrow Q$$

\section{集合}
集合,就是某类事物的聚集在一起;\\
元素,构成集合的各个事物.\\
数学符号为:
$$X = \lbrace 2,4,6 \rbrace\ \text{或}\ X = \lbrace 2n|n=1,2,3 \rbrace$$
若$x$为集合$X$的元素时,表示为:$x \in X$
\subsection{子集}
若集合$X$的所有元素都属于集合$Y$时,那么集合$X$为集合$Y$的子集.\\
数学符号为:
$$X \subset Y$$

\section{映射}
\textbf{从集合$X$到集合$Y$的映射},就是使集合$X$的元素与集合$Y$的元素相对应的规则.\\
数学符号为:
$$X\overrightarrow{\ \ f\ \ }Y\ \text{或}\ f: X\longrightarrow Y$$
注意:映射使得定义域不变时,必定对应相同值域;不会出现定义域不变时,值域产生变化.
\subsection{像}
把通过映射$f$,与集合$X$的元素$x_i$相对应的集合$Y$的元素称为\textbf{$x_i$通过映射$f$形成的像}.\\
数学符号表示为:$$f(x_i)$$
\paragraph{}
事例:
$$f(x)=2x-1$$
意思是
\begin{itemize}
\item 映射$f$是使集合$Y$的元素$2x-1$与集合$X$的元素$x$相对应的规则.
\item 集合$X$的元素$x$通过映射$f$在集合$Y$中形成的像是$2x-1$.
\end{itemize}
\subsection{值域和定义域}
把通过映射$f$形成的像的集合称为\textbf{映射$f$的值域}.\\
与映射$f$的值域相对应的集合$X$称为\textbf{映射$f$的定义域}.
\subsection{满射,单射和满单射}
\paragraph{满射}
映射$f$的值域等于集合$Y$,那么映射$f$是满射,也称为向上映射.
\paragraph{单射}
若$x_i \neq x_j$,则$f(x_i) \neq f(x_j)$,那么映射$f$是单射,也称为一对一的映射.
\paragraph{满单射}
既满足满射,又满足单射则是满单射,也称为双射.
\subsection{逆映射}
若映射$f$和映射$g$满足以下两个条件,那么\textbf{映射$g$是映射$f$的逆映射}.\\
数学表示:
$$X \overrightarrow{\ \ f^{-1}\ \ }Y\ \text{或}\ f^{-1}:X\rightarrow Y$$
定理:
$$\text{与映射f对应的逆映射存在}\Leftrightarrow\text{映射f为满单射}$$
\subsection{线性映射}
假设$x_i$和$x_j$为集合$X$的任意元素,$c$为任意实数,$f$为从集合$X$到集合$Y$的映射.若映射$f$满足以下两个条件,那么\textbf{映射$f$是从集合$X$到集合$Y$的线性映射}.
\begin{itemize}
\item $f(x_i)+f(x_j) = f(x_i+x_j)$
\item $cf(x_i)=f(cx_i)$
\end{itemize}

\section{排列组合}
\paragraph{组合的个数}
从$n$个中挑选$r$个的个数.一般表示为$C_n^r$.
$$C_n^r=\dfrac{n\times(n-1)\times\cdots\times(n-(r-1))}{r\times(r-1)\times\cdots\times 1}$$
$$=\dfrac{n\times(n-1)\times\cdots\times(n-(r-1))}{r\times(r-1)\times\cdots\times 1}\times\dfrac{(n-r)\times(n-(r+1))\times\cdots\times 1}{(n-r)\times(n-(r+1))\times\cdots\times 1}$$
$$=\dfrac{n!}{r!\times(n-r)!}$$
\paragraph{排列的个数}
从$n$个中挑选$r$个事物,然后再将选好的$r$个事物按照顺序排列的种数.一般表示为$P_n^r$.
$$P_n^r=r!\times C_n^r$$

\end{CJK}
\end{document}