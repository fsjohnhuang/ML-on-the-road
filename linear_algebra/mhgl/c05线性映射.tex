\documentclass[12pt]{article}
\usepackage{fullpage}
\usepackage{amsmath}
\usepackage{CJKutf8}

\begin{document}
\begin{CJK}{UTF8}{gbsn}
\section{线性映射}
假设$\forall [x_{1i} \cdots x_{ni}]^T,[x_{1j} \cdots x_{nj}]^T \in R^n$,当满足下面两个条件时,我们称$f$为\textbf{从$R^n$到$R^m$的映射}或\textbf{线性变换}或\textbf{一次变换}.
\begin{itemize}
\item $f\lbrace[x_{1i} \cdots x_{ni}]^T\rbrace + f\lbrace[x_{1j} \cdots x_{nj}]^T\rbrace = f\lbrace[x_{1i}+x{1j} \cdots x_{ni}+x_{nj}]^T\rbrace$
\item $cf\lbrace[x_{1i} \cdots x_{ni}]^T\rbrace = f\lbrace c[x_{1i} \cdots x_{ni}]^T\rbrace$
\end{itemize}
而从$R^n$到$R^m$的映射$f$,实质上就是$M_{m \times n}$的矩阵.
\begin{equation}
\left[
\begin{array}{c}
y_{1}\\
\vdots\\
y_{m}
\end{array}
\right]
=
\left[
\begin{array}{ccc}
a_{11} & \cdots & a_{1n}\\
\vdots & \ddots & \vdots\\
a_{m1} & \cdots & a_{mn}
\end{array}
\right]
\left[
\begin{array}{c}
x_{1}\\
\vdots\\
x_{n}
\end{array}
\right]
\end{equation}
\subsection{用途}
\paragraph{缩放}
假设$[x_1\ y_1]^T \in R^2$,现在沿$X$轴缩放$\alpha$倍,沿$Y$轴缩放$\beta$倍,得到$[x_2\ y_2]^T \in R^2$.
\begin{equation}
\left[
\begin{array}{c}
x_2\\
y_2
\end{array}
\right]
=
\left[
\begin{array}{c}
\alpha x_1\\
\beta y_1
\end{array}
\right]
=
\left[
\begin{array}{cc}
\alpha & 0\\
0 & \beta
\end{array}
\right]
\left[
\begin{array}{c}
x_1\\
y_1
\end{array}
\right]
\end{equation}

\paragraph{旋转}
前提知识
\begin{itemize}
\item $X$轴逆时针转$\theta$度
\begin{equation}
\left[
\begin{array}{c}
x_2\\
y_2
\end{array}
\right]
=
\left[
\begin{array}{cc}
cos\theta & 0\\
sin\theta & 0
\end{array}
\right]
\left[
\begin{array}{c}
x_1\\
0
\end{array}
\right]
\end{equation}
\item $Y$轴逆时针转$\theta$度
\begin{equation}
\left[
\begin{array}{c}
x_2\\
y_2
\end{array}
\right]
=
\left[
\begin{array}{cc}
0 & -sin\theta\\
0 & cos\theta
\end{array}
\right]
\left[
\begin{array}{c}
0\\
y_1
\end{array}
\right]
\end{equation}
\end{itemize}
那么对于$\forall\ [x\ y]^T \in R^2$作旋转可采用以下公式
\begin{equation}
\left[
\begin{array}{c}
x_2\\
y_2
\end{array}
\right]
=
\left[
\begin{array}{cc}
cos\theta & -sin\theta\\
sin\theta & cos\theta
\end{array}
\right]
\left[
\begin{array}{c}
x_1\\
y_1
\end{array}
\right]
\end{equation}

\paragraph{平移}
假设$[x_1\ y_1]^T \in R^2$,现在沿$X$轴平移$\alpha$,沿$Y$轴平移$\beta$,得到$[x_2\ y_2]^T \in R^2$.
\begin{equation}
\left[
\begin{array}{c}
x_2\\
y_2
\end{array}
\right]
=
\left[
\begin{array}{c}
\alpha + x_1\\
\beta + y_1
\end{array}
\right]
=
\left[
\begin{array}{cc}
1 & 0\\
0 & 1
\end{array}
\right]
\left[
\begin{array}{c}
x_1\\
y_1
\end{array}
\right]
+
\left[
\begin{array}{c}
\alpha\\
\beta
\end{array}
\right]
\end{equation}
上述等式并不是线性映射,于是通过\textbf{升维}来转换为线性映射的形式.
\begin{equation}
\left[
\begin{array}{c}
x_2\\
y_2\\
0
\end{array}
\right]
=
\left[
\begin{array}{ccc}
1 & 0 & \alpha\\
0 & 1 & \beta\\
0 & 0 & 1
\end{array}
\right]
\left[
\begin{array}{c}
x_1\\
y_1\\
0
\end{array}
\right]
\end{equation}
对于缩放,旋转,我们也可以通过升维的方式来统一它们的公式.
\paragraph{透视投影}
就是沿着通过一个点的直线,把$3$维空间的点投射为$2$维空间的点.其线性映射$f$固定为
\begin{equation}
\dfrac{1}{x_3-s_3}\left[
\begin{array}{cccc}
-s_3 & 0 & s_1 & 0\\
0 & -s_3 & s_2 & 0\\
0 & 0 & 0 & 0\\
0 & 0 & 1 & -s_3
\end{array}
\right]
\end{equation}
\section{核}
将映射到零向量的所有元素所属的集合称为\textbf{映射$f$的核}.
\begin{equation}
Kerf = \left\lbrace
\left[
\left.
\begin{array}{c}
x_1\\
\vdots\\
x_n
\end{array}
\right]
\right|
\left[
\begin{array}{c}
0\\
\vdots\\
0
\end{array}
\right]=
\left[
\begin{array}{ccc}
a_{11} & \cdots & a_{1n}\\
\vdots & \ddots & \vdots\\
a_{m1} & \cdots & a_{mn}\\
\end{array}
\right]
\left[
\begin{array}{c}
x_1\\
\vdots\\
x_n
\end{array}
\right]
\right\rbrace
\subset R^n
\end{equation}

\section{像空间}
将映射的值域称为\textbf{映射$f$的像空间}.
\begin{equation}
Imf = \left\lbrace
\left[
\left.
\begin{array}{c}
y_1\\
\vdots\\
y_m
\end{array}
\right]
\right|
\left[
\begin{array}{c}
y_1\\
\vdots\\
y_m
\end{array}
\right]=
\left[
\begin{array}{ccc}
a_{11} & \cdots & a_{1n}\\
\vdots & \ddots & \vdots\\
a_{m1} & \cdots & a_{mn}\\
\end{array}
\right]
\left[
\begin{array}{c}
x_1\\
\vdots\\
x_n
\end{array}
\right]
\right\rbrace
\subset R^m
\end{equation}

\section{维数公式}
$Kerf$是$R^n$的子空间,而$Imf$是$R^m$的子空间,它们的维度存在如下关系,称作\textbf{维度公式}$$n-dim\ Kerf = dim\ Imf$$
\subsection{示例}
\paragraph{例1}
\begin{equation}
\text{假设映射}f\left[\begin{array}{cc}
3 & 1\\
1 & 2
\end{array}
\right]\text{是从}R^2\text{到}R^2\text{的线性映射}
\end{equation}
那么求$dim\ Kerf$和$dim\ Imf$
\begin{equation}
Kerf = \left\lbrace
\left[
\left.
\begin{array}{c}
x_1\\x_2
\end{array}
\right]
\right|
\left[
\begin{array}{c}
0\\0
\end{array}
\right]
=
\left[
\begin{array}{cc}
3 & 1\\
1 & 2
\end{array}
\right]
\left[
\begin{array}{c}
x_1\\x_2
\end{array}
\right]
\right\rbrace
=
\left\lbrace
\left[
\begin{array}{c}
0\\0
\end{array}
\right]
\right\rbrace
\end{equation}
由于$Kerf$无法表示为$\lbrace c_1[a_{11}\ \cdots\ a_{m1}]^T + \cdots + c_n[a_{n1}\ \cdots\ a_{mn}]^T\rbrace$,因此$dim\ Kerf=0$.\\
并且$n=2$,所以根据维数公式得到$dim\ Imf=2$.
\paragraph{例2}
\begin{equation}
\text{假设映射}f\left[\begin{array}{cc}
3 & 6\\
1 & 2
\end{array}
\right]\text{是从}R^2\text{到}R^2\text{的线性映射}
\end{equation}
那么求$dim\ Kerf$和$dim\ Imf$
\begin{equation}
Kerf = \left\lbrace
\left[
\left.
\begin{array}{c}
x_1\\x_2
\end{array}
\right]
\right|
\left[
\begin{array}{c}
0\\0
\end{array}
\right]
=
\left[
\begin{array}{cc}
3 & 6\\
1 & 2
\end{array}
\right]
\left[
\begin{array}{c}
x_1\\x_2
\end{array}
\right]
\right\rbrace
=
\left\lbrace
c\left.\left[
\begin{array}{c}
-2\\1
\end{array}
\right]
\right| \forall c \in R
\right\rbrace
\end{equation}
因此$dim\ Kerf=1$,并且$n=2$,根据维数公式得到$dim\ Imf=1$.

\textbf{关于含free variable的augmented matrix求解,要再仔细研究?!}

\section{秩}
\textbf{秩}又称为\textbf{阶数},$R^m$的子空间$Imf$的维数称为\textbf{$m \times n$矩阵的秩}.\\
$m \times n$矩阵就是子空间的基中所有向量元素所组成的矩阵.
\begin{equation}
m \times n\text{矩阵}\left[
\begin{array}{ccc}
a_{11} & \cdots & a_{1n}\\
\vdots & \ddots & \vdots\\
a_{m1} & \cdots & a_{mn}
\end{array}
\right]
\text{的秩一般表示为}
rank\left[
\begin{array}{ccc}
a_{11} & \cdots & a_{1n}\\
\vdots & \ddots & \vdots\\
a_{m1} & \cdots & a_{mn}
\end{array}
\right]
\end{equation}
\subsection{秩的求法}
和gaussian elimination中三类elementary row opertions类似.\\
\textbf{前提:矩阵乘以可逆矩阵不会影响矩阵的秩}.\\
原等式
\begin{equation}
\left[
\begin{array}{ccc}
x_{11} & x_{12} & x_{13}\\
x_{21} & x_{22} & x_{23}\\
x_{31} & x_{32} & x_{33}
\end{array}
\right]
=
\left[
\begin{array}{ccc}
1 & 0 & 0\\
0 & 1 & 0\\
0 & 0 & 1
\end{array}
\right]
\left[
\begin{array}{ccc}
x_{11} & x_{12} & x_{13}\\
x_{21} & x_{22} & x_{23}\\
x_{31} & x_{32} & x_{33}
\end{array}
\right]
\end{equation}
\begin{itemize}
\item[\textbf{行间/列间对调}] 左乘可逆矩阵:第$i$行和第$j$行就会互换位置;右乘可逆矩阵:第$i$列和第$j$列就会互换位置.
\begin{equation}
\left[
\begin{array}{ccc}
x_{31} & x_{32} & x_{33}\\
x_{21} & x_{22} & x_{23}\\
x_{11} & x_{12} & x_{13}
\end{array}
\right]
=
\left[
\begin{array}{ccc}
0 & 0 & 1\\
0 & 1 & 0\\
1 & 0 & 0
\end{array}
\right]
\left[
\begin{array}{ccc}
x_{11} & x_{12} & x_{13}\\
x_{21} & x_{22} & x_{23}\\
x_{31} & x_{32} & x_{33}
\end{array}
\right]
\end{equation}
\begin{equation}
\left[
\begin{array}{ccc}
x_{13} & x_{12} & x_{11}\\
x_{23} & x_{22} & x_{21}\\
x_{33} & x_{32} & x_{31}
\end{array}
\right]
=
\left[
\begin{array}{ccc}
x_{11} & x_{12} & x_{13}\\
x_{21} & x_{22} & x_{23}\\
x_{31} & x_{32} & x_{33}
\end{array}
\right]
\left[
\begin{array}{ccc}
0 & 0 & 1\\
0 & 1 & 0\\
1 & 0 & 0
\end{array}
\right]
\end{equation}
\item[\textbf{行/列缩放}] 左乘可逆矩阵:第$i$行缩放$k$倍;右乘可逆矩阵:第$i$列缩放$k$倍.
\begin{equation}
\left[
\begin{array}{ccc}
x_{11} & x_{12} & x_{13}\\
kx_{21} & kx_{22} & kx_{23}\\
x_{31} & x_{32} & x_{33}
\end{array}
\right]
=
\left[
\begin{array}{ccc}
1 & 0 & 0\\
0 & k & 0\\
0 & 0 & 1
\end{array}
\right]
\left[
\begin{array}{ccc}
x_{11} & x_{12} & x_{13}\\
x_{21} & x_{22} & x_{23}\\
x_{31} & x_{32} & x_{33}
\end{array}
\right]
\end{equation}
\begin{equation}
\left[
\begin{array}{ccc}
x_{11} & kx_{12} & x_{13}\\
x_{21} & kx_{22} & x_{23}\\
x_{31} & kx_{32} & x_{33}
\end{array}
\right]
=
\left[
\begin{array}{ccc}
x_{11} & x_{12} & x_{13}\\
x_{21} & x_{22} & x_{23}\\
x_{31} & x_{32} & x_{33}
\end{array}
\right]
\left[
\begin{array}{ccc}
1 & 0 & 0\\
0 & k & 0\\
0 & 0 & 1
\end{array}
\right]
\end{equation}
\item[\textbf{行/列缩放相加}] 左乘可逆矩阵:第$j$行会加上第$i$行的$k$倍;右乘可逆矩阵:第$i$列加上第$j$列的$k$倍.
\begin{equation}
\left[
\begin{array}{ccc}
x_{11} & x_{12} & x_{13}\\
x_{21} & x_{22} & x_{23}\\
kx_{21}+x_{31} & kx_{22}+x_{32} & kx_{23}+x_{33}
\end{array}
\right]
=
\left[
\begin{array}{ccc}
1 & 0 & 0\\
0 & 1 & 0\\
0 & k & 1
\end{array}
\right]
\left[
\begin{array}{ccc}
x_{11} & x_{12} & x_{13}\\
x_{21} & x_{22} & x_{23}\\
x_{31} & x_{32} & x_{33}
\end{array}
\right]
\end{equation}
\begin{equation}
\left[
\begin{array}{ccc}
x_{11} & x_{12}+kx_{13} & x_{13}\\
x_{21} & x_{22}+kx_{23} & x_{23}\\
x_{31} & x_{32}+kx_{33} & x_{33}
\end{array}
\right]
=
\left[
\begin{array}{ccc}
x_{11} & x_{12} & x_{13}\\
x_{21} & x_{22} & x_{23}\\
x_{31} & x_{32} & x_{33}
\end{array}
\right]
\left[
\begin{array}{ccc}
1 & 0 & 0\\
0 & 1 & 0\\
0 & k & 1
\end{array}
\right]
\end{equation}
\end{itemize}

\section{小结}
由于直接求子空间的维度并不容易,而我们手头上一般拥有的是映射$f$对应的$m \times n$矩阵,因此可通过两种方式间接解决
\begin{enumerate}
\item 先求映射$f$核$Kerf$的维度,再通过维数公式求映射$f$像空间$Imf$(即子空间)的维度.
\item 求映射$f$像空间$Imf$的对应的秩
\end{enumerate}

\end{CJK}
\end{document}