\documentclass[12pt]{article}
\usepackage{amsmath}
\usepackage{fullpage}
\usepackage{CJKutf8}

\begin{document}
\begin{CJK}{UTF8}{gbsn}
\section{矩阵}
用处清晰表示一次方程组.
\subsection{乘法}
$m \times n$矩阵与$n \times p$矩阵相乘:
\begin{equation}
\left[
\begin{array}{ccc}
a_{11} & \cdots & a_{1n} \\
\vdots & \ddots & \vdots \\
a_{m1} & \cdots & a_{mn} 
\end{array}
\right]
\left[
\begin{array}{ccc}
b_{11} & \cdots & b_{1p} \\
\vdots & \ddots & \vdots \\
b_{n1} & \cdots & b_{np} 
\end{array}
\right]
\end{equation}
\begin{equation}
=
\left[
\begin{array}{ccc}
a_{11} & \cdots & a_{1n} \\
\vdots & \ddots & \vdots \\
a_{m1} & \cdots & a_{mn} 
\end{array}
\right]
\left[
\begin{array}{c}
b_{11} \\
\vdots \\
b_{n1}
\end{array}
\right]
\cdots
\left[
\begin{array}{ccc}
a_{11} & \cdots & a_{1n} \\
\vdots & \ddots & \vdots \\
a_{m1} & \cdots & a_{mn} 
\end{array}
\right]
\left[
\begin{array}{c}
b_{1p} \\
\vdots \\
b_{np}
\end{array}
\right]
\end{equation}
\begin{equation}
=\left[
\begin{array}{ccc}
a_{11}\times b_{11}+\cdots+a_{1n}\times b_{n1} & \cdots & a_{11}\times b_{1p}+\cdots+a_{1n}\times b_{np} \\
\vdots & \ddots & \vdots \\
a_{m1}\times b_{11}+\cdots+a_{mn}\times b_{n1} & \cdots & a_{m1}\times b_{1p}+\cdots+a_{mn}\times b_{np}
\end{array}
\right]
\end{equation}
\begin{itemize}
\item 矩阵乘法拆解为矩阵向量乘法
\item 只有当左边的矩阵的列数等于右边矩阵的行数时,两个矩阵才可以进行乘法运算
\item 矩阵相乘不遵循乘法交换律
\end{itemize}

\subsection{次方}
\begin{equation}
\left[
\begin{array}{ccc}
a_{11} & \cdots & a_{1n} \\
\vdots & \ddots & \vdots \\
a_{m1} & \cdots & a_{mn} 
\end{array}
\right]^n
\end{equation}

\subsection{转置矩阵}
\begin{equation}
\text{转置矩阵是指}m \times n\text{矩阵}
\left[
\begin{array}{ccc}
a_{11} & \cdots & a_{1n} \\
\vdots & \ddots & \vdots \\
a_{m1} & \cdots & a_{mn} 
\end{array}
\right]
\text{的行和列交换后得到的}n \times m
\text{矩阵}\left[
\begin{array}{ccc}
a_{11} & \cdots & a_{m1} \\
\vdots & \ddots & \vdots \\
a_{1n} & \cdots & a_{mn} 
\end{array}
\right]\text{.}
\end{equation}
\begin{equation}
m \times n\text{矩阵}\left[
\begin{array}{ccc}
a_{11} & \cdots & a_{1n} \\
\vdots & \ddots & \vdots \\
a_{m1} & \cdots & a_{mn} 
\end{array}
\right]\text{的转置矩阵可以表示为}\left[
\begin{array}{ccc}
a_{11} & \cdots & a_{1n} \\
\vdots & \ddots & \vdots \\
a_{m1} & \cdots & a_{mn} 
\end{array}
\right]^T
\end{equation}

\subsection{对角矩阵}
对角矩阵就是除对角元素外的元素均为$0$的$n$阶方阵.如下对角矩阵可表示为$diag(1,2,3,4)$,diag是对角线diagonal的缩写.
\begin{equation}
\left[
\begin{array}{cccc}
1 & 0 & 0 & 0 \\
0 & 2 & 0 & 0 \\
0 & 0 & 3 & 0 \\
0 & 0 & 0 & 4 \\
\end{array}
\right]
\end{equation}

\subsection{单位矩阵}
单位矩阵就是对角元素均为$1$,其余元素均为$0$的$n$阶方阵,也就是$diag(1,\cdots,1)$.
\paragraph{}
单位矩阵与其他矩阵或向量相乘,不会产生任何变化.

\section{逆矩阵}
与$n$阶方阵$A$相乘的积等于单位矩阵的$n$阶方阵$B$,就是$A$的逆矩阵,表示为$A^{-1}$.即
\begin{equation}
\left[
\begin{array}{ccc}
a_{11} & \cdots & a_{1n} \\
\vdots & \ddots & \vdots \\
a_{n1} & \cdots & a_{nn} 
\end{array}
\right]
\left[
\begin{array}{ccc}
x_{11} & \cdots & x_{1n} \\
\vdots & \ddots & \vdots \\
x_{n1} & \cdots & x_{nn} 
\end{array}
\right]
=
\left[
\begin{array}{ccc}
1 & \cdots & 0 \\
\vdots & \ddots & \vdots \\
0 & \cdots & 1
\end{array}
\right]
\end{equation}
由于矩阵$A$存在对应的逆矩阵$B$,因此矩阵$A$被称为\textbf{可逆矩阵}.(并不是所有矩阵均为可逆矩阵)\\
\textbf{"原来的矩阵"与"逆矩阵"相乘,遵循乘法交换律.}
\subsection{逆矩阵的求解方式-高斯消元法}
\paragraph{事例}
\begin{equation}
\left[
\begin{array}{cc}
3 & 1 \\
1 & 2
\end{array}
\right]
\left[
\begin{array}{cc}
x_{11} & x_{12} \\
x_{21} & x_{22}
\end{array}
\right]
=
\left[
\begin{array}{cc}
1 & 0 \\
0 & 1
\end{array}
\right]
\end{equation}
求逆矩阵.
解:
\begin{equation}
\text{分解为}
\left[
\begin{array}{cc}
3 & 1 \\
1 & 2
\end{array}
\right]
\left[
\begin{array}{c}
x_{11} \\
x_{21}
\end{array}
\right]
=
\left[
\begin{array}{cc}
1 \\
0
\end{array}
\right]
\text{和}
\left[
\begin{array}{cc}
3 & 1 \\
1 & 2
\end{array}
\right]
\left[
\begin{array}{c}
x_{12} \\
x_{22}
\end{array}
\right]
=
\left[
\begin{array}{cc}
0 \\
1
\end{array}
\right]
\end{equation}
\begin{equation}
\text{转换为augmented matrix}
\left[
\begin{array}{ccc}
3 & 1 & 1\\
1 & 2 & 0
\end{array}
\right]
\text{和}
\left[
\begin{array}{ccc}
3 & 1 & 0\\
1 & 2 & 1
\end{array}
\right]
\end{equation}
\begin{equation}
\text{合并}\left[
\begin{array}{cccc}
3 & 1 & 1 & 0 \\
1 & 2 & 0 & 1
\end{array}
\right]
\end{equation}
\begin{equation}
\text{高斯消元法后}\left[
\begin{array}{cccc}
1 & 0 & \dfrac{2}{5} & -\dfrac{1}{5} \\
0 & 1 & -\dfrac{1}{5} & \dfrac{3}{5}
\end{array}
\right]
\end{equation}
\begin{equation}
\text{逆矩阵为}\left[
\begin{array}{cc}
\dfrac{2}{5} & -\dfrac{1}{5} \\
-\dfrac{1}{5} & \dfrac{3}{5}
\end{array}
\right]
\end{equation}
\subsection{2阶矩阵的逆矩阵求解方式}
\begin{equation}
\left[
\begin{array}{cc}
a_{11} & a_{12}\\
a_{21} & a_{22}
\end{array}
\right]^{-1}
=
\dfrac{1}{a_{11}a_{22}-a_{12}a_{21}}
\left[
\begin{array}{cc}
a_{22} & -a_{12}\\
-a_{21} & a_{11}
\end{array}
\right]
\end{equation}

\subsection{行列式}
用于确定$n$阶矩阵是否为可逆矩阵.
\begin{equation}
\text{矩阵}\left[
\begin{array}{ccc}
x_{11} & \cdots & x_{1n} \\
\vdots & \ddots & \vdots \\
x_{n1} & \cdots & x_{nn}
\end{array}
\right]\text{的行列式表示为}
det\left[
\begin{array}{ccc}
x_{11} & \cdots & x_{1n} \\
\vdots & \ddots & \vdots \\
x_{n1} & \cdots & x_{nn}
\end{array}
\right] \text{或}
\left|
\begin{array}{ccc}
x_{11} & \cdots & x_{1n} \\
\vdots & \ddots & \vdots \\
x_{n1} & \cdots & x_{nn}
\end{array}
\right|
\end{equation}
$det$是determinant(决定因子)的缩写.
\paragraph{存在可逆矩阵的条件}
\begin{equation}
\left|
\begin{array}{ccc}
x_{11} & \cdots & x_{1n} \\
\vdots & \ddots & \vdots \\
x_{n1} & \cdots & x_{nn}
\end{array}
\right| \neq 0 \Rightarrow 
\left[
\begin{array}{ccc}
x_{11} & \cdots & x_{1n} \\
\vdots & \ddots & \vdots \\
x_{n1} & \cdots & x_{nn}
\end{array}
\right]^{-1}\text{存在.}
\end{equation}
\paragraph{行列式的计算}
\begin{equation}
\left|
\begin{array}{ccc}
x_{11} & \cdots & x_{1n} \\
\vdots & \ddots & \vdots \\
x_{n1} & \cdots & x_{nn}
\end{array}
\right|
\end{equation}
$$
=x_{11}x_{22}\cdots x_{nn} + x_{12}x_{23}\cdots x_{n1} + \cdots +x_{1n}x_{21}\cdots x_{n(n-1)}
$$
$$
- x_{1n}x_{2(n-1)}\cdots x_{n1}
- x_{1(n-1)}x_{2(n-2)}\cdots x_{nn}
\cdots
- x_{11}x_{2n}\cdots x_{n2}
$$


\end{CJK}
\end{document}
