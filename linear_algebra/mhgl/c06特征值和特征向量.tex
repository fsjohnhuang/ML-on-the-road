\documentclass[12pt]{article}
\usepackage{fullpage}
\usepackage{amsmath}
\usepackage{CJKutf8}

\begin{document}
\begin{CJK}{UTF8}{gbsn}
\section{特征值和特征向量}
\subsection{定义}
\begin{equation}
\text{当向量}
\left[
\begin{array}{c}
x_1\\
\vdots\\
x_n
\end{array}
\right]
\text{通过"}n\text{阶方阵}
\left[
\begin{array}{ccc}
x_{11} & \cdots & x_{1n}\\
\vdots & \ddots & \vdots\\
x_{n1} & \cdots & x_{nn}
\end{array}
\right]
\text{对应的线性映射}f\text{"形成的像为}
\lambda\left[
\begin{array}{c}
x_1\\
\vdots\\
x_n
\end{array}
\right]
\text{时,} 
\end{equation}
那么$\lambda$称为$n$阶方阵的\textbf{特征值},向量$[x_1\cdots x_n]^T$称为$\lambda$的特征向量.而零向量不能作为特征向量.\\
\textbf{注意:}特征值是指某映射的特征值,特征向量是指某映射下某特征值的特征向量.
\subsection{求特征值和特征向量}
\paragraph{特征值求法}
特征值与映射对应的行列式有如下的关系:
\begin{equation}
\lambda\text{为}n\text{阶方阵}\left[
\begin{array}{ccc}
x_{11} & \cdots & x_{1n}\\
\vdots & \ddots & \vdots\\
x_{n1} & \cdots & x_{nn}
\end{array}
\right]\text{的特征值}
\Leftrightarrow
det
\left[
\begin{array}{ccc}
x_{11}-\lambda & \cdots & x_{1n}\\
\vdots & \ddots & \vdots\\
x_{n1} & \cdots & x_{nn}-\lambda
\end{array}
\right]=0
\end{equation}
\paragraph{特征向量求法}
得到特征值后只需带入映射公式即可得特征向量.示例:
\begin{equation}
\left[
\begin{array}{cc}
8 & -3\\
2 & 1
\end{array}
\right]
\left[
\begin{array}{c}
x_1\\
x_2
\end{array}
\right]=\lambda
\left[
\begin{array}{c}
x_1\\
x_2
\end{array}
\right]
\end{equation}
\begin{equation}
\left[
\begin{array}{cc}
8 & -3\\
2 & 1
\end{array}
\right]
\left[
\begin{array}{c}
x_1\\
x_2
\end{array}
\right]=
\left[
\begin{array}{cc}
\lambda & 0\\
0 & \lambda
\end{array}
\right]
\left[
\begin{array}{c}
x_1\\
x_2
\end{array}
\right]
\end{equation}
\begin{equation}
\left[
\begin{array}{cc}
8 & -3\\
2 & 1
\end{array}
\right]
\left[
\begin{array}{c}
x_1\\
x_2
\end{array}
\right]-
\left[
\begin{array}{cc}
\lambda & 0\\
0 & \lambda
\end{array}
\right]
\left[
\begin{array}{c}
x_1\\
x_2
\end{array}
\right]=\left[
\begin{array}{c}
0\\
0
\end{array}
\right]
\end{equation}
\begin{equation}
\left(\left[
\begin{array}{cc}
8 & -3\\
2 & 1
\end{array}
\right]
-
\left[
\begin{array}{cc}
\lambda & 0\\
0 & \lambda
\end{array}
\right]\right)
\left[
\begin{array}{c}
x_1\\
x_2
\end{array}
\right]=\left[
\begin{array}{c}
0\\
0
\end{array}
\right]
\end{equation}
\begin{equation}
\left[
\begin{array}{cc}
8-\lambda & -3\\
2 & 1-\lambda
\end{array}
\right]
\left[
\begin{array}{c}
x_1\\
x_2
\end{array}
\right]=\left[
\begin{array}{c}
0\\
0
\end{array}
\right]
\end{equation}
\section{$n$阶方阵$p$次幂的求法}
由于$n$阶方阵特征值和特征向量关系如下:
\begin{equation}
\left[
\begin{array}{ccc}
x_{11} & \cdots & x_{1n}\\
\vdots & \ddots & \vdots\\
x_{n1} & \cdots & x_{nn}
\end{array}
\right]
\left[
\begin{array}{c}
a_1\\
\vdots\\
a_n
\end{array}
\right]
=\lambda
\left[
\begin{array}{c}
a_1\\
\vdots\\
a_n
\end{array}
\right]
\end{equation}
然后将$n$阶方阵所有特征值和特征向量合并为如下公式:
\begin{equation}
\left[
\begin{array}{ccc}
x_{11} & \cdots & x_{1n}\\
\vdots & \ddots & \vdots\\
x_{n1} & \cdots & x_{nn}
\end{array}
\right]
\left[
\begin{array}{ccc}
a_{11} & \cdots & a_{1n}\\
\vdots & \ddots & \vdots\\
a_{n1} & \cdots & a_{nn}
\end{array}
\right]
=
\left[
\begin{array}{ccc}
\lambda_1a_{11} & \cdots & \lambda_na_{1n}\\
\vdots & \ddots & \vdots\\
\lambda_1a_{n1} & \cdots & \lambda_na_{nn}
\end{array}
\right]
\end{equation}
\begin{equation}
=
\left[
\begin{array}{ccc}
a_{11} & \cdots & a_{1n}\\
\vdots & \ddots & \vdots\\
a_{n1} & \cdots & a_{nn}
\end{array}
\right]
\left[
\begin{array}{ccc}
\lambda_1 & \cdots & 0\\
\vdots & \ddots & \vdots\\
0 & \cdots & \lambda_n
\end{array}
\right]
\end{equation}
然后通过两边乘以逆矩阵得到
\begin{equation}
\left[
\begin{array}{ccc}
x_{11} & \cdots & x_{1n}\\
\vdots & \ddots & \vdots\\
x_{n1} & \cdots & x_{nn}
\end{array}
\right]
=
\left[
\begin{array}{ccc}
a_{11} & \cdots & a_{1n}\\
\vdots & \ddots & \vdots\\
a_{n1} & \cdots & a_{nn}
\end{array}
\right]
\left[
\begin{array}{ccc}
\lambda_1 & \cdots & 0\\
\vdots & \ddots & \vdots\\
0 & \cdots & \lambda_n
\end{array}
\right]
\left[
\begin{array}{ccc}
a_{11} & \cdots & a_{1n}\\
\vdots & \ddots & \vdots\\
a_{n1} & \cdots & a_{nn}
\end{array}
\right]^{-1}
\end{equation}
那么$n$阶方阵的$p$次幂则是
\begin{equation}
\left[
\begin{array}{ccc}
x_{11} & \cdots & x_{1n}\\
\vdots & \ddots & \vdots\\
x_{n1} & \cdots & x_{nn}
\end{array}
\right]^p
=
\left[
\begin{array}{ccc}
a_{11} & \cdots & a_{1n}\\
\vdots & \ddots & \vdots\\
a_{n1} & \cdots & a_{nn}
\end{array}
\right]
\left[
\begin{array}{ccc}
\lambda_1^p & \cdots & 0\\
\vdots & \ddots & \vdots\\
0 & \cdots & \lambda_n^p
\end{array}
\right]
\left[
\begin{array}{ccc}
a_{11} & \cdots & a_{1n}\\
\vdots & \ddots & \vdots\\
a_{n1} & \cdots & a_{nn}
\end{array}
\right]^{-1}
\end{equation}
\section{是否存在重解和对角化?}
并不是所有$n$阶方阵都可以以它的特征值矩阵和特征向量矩阵表示,这取决于特征值对应的特征向量是否足够.\\
足够的情况:
\begin{equation}
n\text{阶方阵}\left[
\begin{array}{ccc}
1 & 0 & 0 \\
1 & 1 & -1 \\
-2 & 0 & 3
\end{array}
\right]\text{的特征值分别是}3\text{和}1\text{,}
\end{equation}
\begin{equation}
\text{然后分别求出它们对应的特征向量}
c_1\left[
\begin{array}{c}
0\\1\\-2
\end{array}
\right]
\text{和}
c_2\left[
\begin{array}{c}
1\\0\\1
\end{array}
\right]
+
c_3\left[
\begin{array}{c}
0\\1\\0
\end{array}
\right]
\end{equation}
\begin{equation}
\text{最后得出}
\left[
\begin{array}{ccc}
0 & 1 & 0\\
1 & 0 & 1\\
-2 & 1 & 0
\end{array}
\right]
\left[
\begin{array}{ccc}
3 & 0 & 0\\
0 & 1 & 0\\
0 & 0 & 1
\end{array}
\right]
\left[
\begin{array}{ccc}
0 & 1 & 0\\
1 & 0 & 1\\
-2 & 1 & 0
\end{array}
\right]^{-1}
\end{equation}
若$1$对应的特征向量无法表示为$c_2[1\ 0\ 1]^T+c_3[0\ 1\ 0]^T$的形式则无法得到最后一条式子.

\end{CJK}
\end{document}