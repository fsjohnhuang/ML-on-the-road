\documentclass[12pt]{article}
\usepackage{fullpage}
\usepackage{amssymb}

\begin{document}
\tableofcontents
\title{Chapter1 Basic Concepts on Matrices and Vectors}
\author{fsjohnhuang}
\date{\today}
\maketitle

\section{Matrix}
\subsection{Definitions}
A \textbf{matrix} is a rectangular array of scalars.\\
If the matrix has $m$ rows and $n$ columns, we say the \textbf{size} of matrix is \textbf{m by n}, written $m \times n$.\\
The matrix is called square if $m = n$.\\
The scalar in $i$th row and $j$th column is called \textbf{$(i,j)$-entry} of the matrix.\\

\subsection{Notation}
\begin{equation}
	A =
	\left[
	\begin{array}{ccc}
	a_{11} & \cdots  & a_{1n} \\
	\vdots & \ddots & \vdots \\
	a_{m1} & \cdots  & a_{mn}
	\end{array}
	\right]
	= [a_{ij}] \in M_{m \times n}
\end{equation}
$M_{m \times n}$ denotes the set that contains all matrices whose size is $m \times n$.\\

\subsection{Equality of matrices}
\subsubsection{Definitions}
We say two matrices $A$ and $B$ are equal if
\begin{enumerate}
\item they have the same size.
\item they have equal corresponding entries.
\end{enumerate} 
Let $A,B \in M_{m \times n}$

Then $A = B \Longleftrightarrow a_{ij} = b_{ij}, \forall i=1, \ldots ,m, j=1, \ldots ,n$

\subsection{Submatrices}
\subsubsection{Definitions}
A submatrix is obtained by deleting from a matrix entire rows and/or columns.\\
\subsubsection{Sample}
\begin{equation}
	\left[
	\begin{array}{ccc}
	1 & 2 & 3 \\
	2 & 1 & 4 \\
	\end{array}
	\right]
	is\ a\ submatrix\ of
	\left[
	\begin{array}{ccc}
	1 & 2 & 3 \\
	2 & 1 & 4 \\
	1 & 3 & 2 \\
	\end{array}
	\right]
\end{equation}

\subsection{Matrix addition}
\subsubsection{Definitions}
Let A and B be in matrix $m \times n$. We define the sum of A and B, denoted $A+B$, to be a $m \times n$ matrix obtained by adding the corressponding entries of A and B; that is the $m \times n$ matrix whose $(i,j)-entry$ is $a_{ij}+b{ij}$.
\subsubsection{Notation}
Let $A,B \in M_{m \times n}$

Then $A+B=[a_{ij}+b_{ij}], \forall i=1, \ldots ,m, j=1, \ldots ,n$
\subsubsection{Theorem}
\begin{enumerate}
\item[Commutative] $A+B=B+A$
\item[Associative] $(A+B)+C=A+(B+C)$
\end{enumerate}

\subsection{Scalar Multiplication}
\subsubsection{Definitions}
Let $A$ be an $m \times n$ matrix and $c$ be a scalar. The scalar multiple $cA$ of matrix $A$ is defined to be the $m \times n$ matrix whose $(i,j)-entries$ is $c \times a_{ij}$.
\subsubsection{Notation}
Let $A \in M_{m \times n}$ and $c \in R$

Then $cA=[c \times a_{ij}], \forall i=1, \ldots ,m, j=1, \ldots ,n$
\subsubsection{Theorem}
\begin{enumerate}
\item[Associative] $(st)A=s(tA)\ ,(s,t \in R)$
\item[Distributive] $s(A+B)=sA+sB$ or $(s+t)A=sA+tA$ ,$(s \in R)$
\end{enumerate}

\subsection{Zero Matrices}
\subsubsection{Definitions}
A \textbf{zero matrix} with all zero entries, denoted by $O$(any size) or $O_{m \times n}$.
\subsubsection{Properties}
\begin{enumerate}
\item $A = O + A, \forall A \in M_{m \times n}$
\item $0 \cdot A = O, \forall A \in M_{m \times n}$
\end{enumerate}

\subsection{Matrix Substraction}
\subsubsection{Definitions}
We define the matrix $-A$ to be $-1(A)$. The matrix substraction of the two matrix $A$ and $B$ is define to be as
$$A-B = A+(-B)$$
\subsubsection{Theorem}
\begin{enumerate}
\item $A-A = O, \forall A \in M_{m \times n}$
\end{enumerate}

\subsection{Matrix Transpose}
\subsubsection{Definitions}
The transpose of a $m \times n$ matrix $A$ is the $n \times m$ matrix $A^T$ whose $(i,j)-enrty$ is the $(j,i)-enrty$ of $A.$ 
\subsubsection{Properties}
$$A \in M_{m \times n} \Rightarrow A^T \in M_{n \times m}$$
\subsubsection{Theorem}
\begin{enumerate}
\item[associative] $(sA)^T=s(A^T)\ ,\forall s \in R$
\item[Distributive] $(A+B)^T=A^T+B^T$
\item[] $(A^T)^T=A$
\end{enumerate}

\section{Vector}
\subsection{Definitions}
Vector can refer to either a \textbf{row vector} or a \textbf{column vector}.\\
Row vector is a matrix with \textbf{one} row.
$$\underline{v}=[a_1,\cdots,a_n]$$
Column vector is a matrix with \textbf{one} column.
\begin{equation}
	\underline{v}=
	\left[
	\begin{array}{c}
	a_1 \\
	\vdots \\
	a_m
	\end{array}
	\right]
	\ or\ \underline{v}^T=[a_1,\cdots,a_m]
\end{equation}
So row vector can transpose to column vector, vice versa.\\
\subsection{Notation}
\begin{equation}
	\underline{v}=
	\left[
	\begin{array}{c}
	a_1 \\
	\vdots \\
	a_n
	\end{array}
	\right]
	= [a_i] \in R^n , \forall i = 1,\cdots,n
\end{equation}
We denote the set of all column vectors with $n$ components by $R^n$. That is $R^n = M_{n \times 1}.$

\subsection{Vector Addition/Substraction, Scalar Multiplication and Zero Vector}
Follow those for matrices.\\

Zero Vector is denoted \begin{large}$\textbf{0}$\end{large}\\\\
A matrix is often regarded as a stack of row vectors or a cross list of column vectors.\\
Let $C \in M_{m \times n}$\\
$$C = [\underline{c_1},\ \cdots\ ,\underline{c_j},\ \cdots\ ,\underline{c_n}]$$
\begin{equation}
	\underline{c_j}=
	\left[
	\begin{array}{c}
	c_{1j} \\
	\vdots \\
	c_{mj}
	\end{array}
	\right]
	= [c_i] \in R^m , \forall i = 1,\cdots,m
\end{equation}

\section{Linear Combinations}
\subsection{Definitions}
A linear combination of vectors $\underline{u_1},\ \cdots\ , \underline{u_n}$ is a vector of the form
$$\underline{u}=c_1\underline{u_1} + \cdots + c_n\underline{u_n}$$
Where $c_1\ ,\cdots\ ,c_n$ are scalars which called \textbf{coefficients} of the linear combination.
\subsection{Samples}
\subsubsection{Given coefficients, computes linear combination}
Let $\underline{u_1}^T=[-1,-3,4]$, $\underline{u_2}^T=[-4,1,2]$, $c_1=3$ and $c_2=2$

\begin{equation}
Then\ 
c_1\underline{u_1}^T + c_2\underline{u_2}^T = 
\left[
\begin{array}{ccc}
-3 & -9 & 12
\end{array}
\right] +
\left[
\begin{array}{ccc}
-12 & 3 & 6
\end{array}
\right]
= 
\left[
\begin{array}{ccc}
-15 & -6 & 18
\end{array}
\right]
\end{equation}
\begin{equation}
\because\ c_1\underline{u_1}^T + c_2\underline{u_2}^T = (c_1\underline{u_1} + c_2\underline{u_2})^T
\end{equation}
\begin{equation}
\therefore\ c_1\underline{u_1} + c_2\underline{u_2} = 
\left[
\begin{array}{c}
-15 \\
-6 \\
18
\end{array}
\right]
\end{equation}
\subsubsection{Given linear combination, computes coefficients}
Which could be transform to solve a system of linear equations. But there are three solution.\\
\begin{enumerate}
\item[] Unique solution when $s\underline{c_1}$ and $\underline{c_2}$ are not \textbf{collinear vectors}.
\item[] Infinitely many solutions when $s\underline{c_1},\ \underline{c_2}$ are \textbf{collinear vectors} and $\underline{c}=s\underline{c_1},\ \underline{c}$ is linear combination
\item[] No solutions when  $s\underline{c_1},\ \underline{c_2}$ are \textbf{collinear vectors} and $\underline{c}\neq s\underline{c_1},\ \underline{c}$ is linear combination
\end{enumerate}

\subsection{Parallel/Collinear Vectors}
Let $\underline{a}$ and $\underline{b}$ not be in zero vector. We define $\underline{a}$ is parallel with $\underline{b}$ when $s\underline{a} = \underline{b}\ ,s \in R$.\\
\subsubsection{Notations}
Let $\forall \underline{a},\underline{b} \in R^n$\\

Then $\underline{a} \parallel \underline{b} \Rightarrow s\underline{a} = \underline{b}\ ,s \in R$\\\\
$\underline{a} \parallel \underline{b}$ denotes the vectors $\underline{a}$, $\underline{b}$ are the parallel or collinear vector\\

\section{Standard Vectors}
\subsection{Definitions}
The standard vectors of $R^n$ are defined as
\begin{equation}
\underline{e_1} = \left[
\begin{array}{c}
1\\
0\\
\vdots\\
0
\end{array}
\right]\ 
\underline{e_2} = \left[
\begin{array}{c}
0\\
1\\
\vdots\\
0
\end{array}
\right]\ \cdots \ 
\underline{e_n} = \left[
\begin{array}{c}
0\\
0\\
\vdots\\
1
\end{array}
\right]\ 
\end{equation}
\subsection{Properties}
Every vector in $R^n$ may be uniquely linearly combined by standard vectors.

\section{Matrix-Vector Product}
\subsection{Definitions}
Let $A$ be an $m \times n$ matrix and $\underline{v}$ be an $n \times 1$ vector. We define the \textbf{matrix-vector product} of $A$ and $\underline{v}$, denoted by $A\underline{v}$, to be the linear combination of the columns of $A$ whose coefficients are the corresponding components of $\underline{v}$. That is 
$$A\underline{v} = v_1\underline{a_1} + v_2\underline{a_2} + \cdots + v_n\underline{a_n}$$
\subsection{Cautions}
\begin{enumerate}
\item the size of $\underline{v}$ must be equal to the count of columns of $A$.
\item the solution of matrix-vector product is a new vector whose size is same with the count of rows of $A$.
\end{enumerate}
\subsection{Theorems}
\begin{enumerate}
\item[Distributive] $A(\underline{u}+\underline{v})=A\underline{u}+A\underline{v}$ and $(A+B)\underline{u}=A\underline{u}+B\underline{u}$
\item[Associative] $A(c\underline{u})=c(A\underline{u})=(cA)\underline{u},\ \forall\ c \in R$
\item[] $A\underline{0}=\underline{0}\ and\ O\underline{v}=\underline{0}\ , \forall A\ \in M_{m \times n}\ and\ \underline{v} \in R^n$
\item[] $A\underline{e_j} = \underline{a_j}\ ,\ A = [\underline{a_1}\ \cdots\ \underline{a_m}]$
\item[] $\ B\underline{w}=A\underline{w} \Rightarrow A=B\ ,\ \forall\ B,A \in M_{m \times n}\ and\ \forall\ \underline{w} \in R^n.$ 
\end{enumerate}

\section{Identity Matrix}
\subsection{Definitions}
For each positive integer $n$, the $n \times n$ \textbf{identity matrix} $I_n$ is the  $n \times n$ matrix whose respective columns are the standard vectors $\underline{e_1},\ \underline{e_2},\ \cdots\ ,\underline{e_n}$ in $R^n$\\
Sometime $I_n$ is simply written as $I$.
\subsection{Sample}
\begin{equation}
I_4 = \left[
\begin{array}{cccc}
1 & 0 & 0 & 0\\
0 & 1 & 0 & 0\\
0 & 0 & 1 & 0\\
0 & 0 & 0 & 1
\end{array}
\right]
\end{equation}
\subsection{Properties}
$$I_n\underline{v} = \underline{v}\ ,\forall\ \underline{v} \in R^n$$

\section{Stochastic Matrix}
\subsection{Definitions}
An matrix $A \in M_{m \times n}$ is called a stochastic matrix if all entries of $A$ is nonnegative and the sum of all entries in each columns is unity.

\end{document}